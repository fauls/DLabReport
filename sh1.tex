\newpage
\section{Описание прототипа программного продукта}
Для создания депозитного калькулятора была выбрана среда выполнения Eclipce;
\\Для работы с визуальными формами было установлено официальное расширение
WindowBuilder, которое добавляет визуальную форму в которой можно редактировать
Swing компоненты.\\
\\Были определены входные и выходные данные для депозитного калькулятора:
\begin{itemize}
\item Сумма вклада
\item Дата открытия
\item Процентная ставка по вкладу
\item Срок вклада
\item Капитализация вклада
\item Периодичность капитализации
\end{itemize}
Из выходных данных требовались:
\begin{itemize}
	\item Итоговый депозит
	\item Отдельно количество процентов
	\item Опционально таблица с периодами выплат
\end{itemize}
После анализа входных и выходных данных была собрана визуальная форма в WindowBuilder.
\imgh{150mm}{screenshot001}{Итоговый вид формы}
Форма состоит из полей ввода для информации, чекбокса для выбора капитализации
и комбобокса для выбора частоты капитализации. 
Для того, что бы без выбора капитализации нельзя было сменить его частоту был 
использован следующий код:
\newpage
\lstinputlisting[language=Java, caption=Нажатие на флажок]{first.java}
Так же форма содержит место для скроллящейся таблицы.
\section{Программный код}
Проект состоит из 4х файлов:
\begin{itemize}
	\item Deposit.java - основной класс, получает входные данные
	\item Proc.java - содержит энумератор для комбобокса
	\item Test.java - содержит юнит-тесты
	\item wind.java - основной класс UI и обработки данных
\end{itemize}
Deposit.java содержит методы
\begin{itemize}
	\item public void calculate() - публичный метод, обрабатывает данные
	\item public void setDate(int date) - проверяет входит ли день в месяц, в случае ошибки (например 29е число в феврале) генерирует исключение.
	\item private void dayInPeriod() - считает время от от начала вклада до забора
	\item private double howDay() - возвращает кол-во дней в текущем месяце
	\item private double mathMoney() - подсчитывает проценты по формуле
	\item private void addCapital(double capMoney, int i) - добавляет проценты к общей сумме и вносит строку в таблицу
	\item private void simpleInterest() - подсчитывает простые проценты
	\item private void hardInterest() - подсчитывает сложные проценты
\end{itemize}
Wind.java содержит методы:
\begin{itemize}
	\item private void initComp() - создание GUI
	\item private void create\_event() - обработка событий
	\begin{itemize}
		\item Нажатие на флажок - блокирует комбобокс
		\item Нажатие на кнопку расчёта - проверяет данные из текстбоксов, в случае успеха передает их созданному экземпляру класса deposit;
	\end{itemize}
\end{itemize}
Полный исходный код всех файлов приведен в приложении.
Так же весь исходный код доступен в репозитории github:
https://github.com/fauls/DCalc
\imgh{150mm}{screenshot003}{Страница репозитория}

\section{Описание тестирования работы программы.}
Для тестирования использовались JUnit5 тесты, содержащиеся в файле Test.java.
\\В ходе тестирование создавался экземпляр класса deposit, которому передавались стартовые значения и вызывался основной метод calculate.После этого тестовой функцией assertEquals проводилось сравнение выхода программы и эталонного значения, полученного с различных онлайн-депозитных калькуляторов.\\
\\Всего было написано 8 тестовых кейсов с различными условиями, полный код юнит-тестов можно найти в приложении.
\imgh{150mm}{screenshot002}{Пример пройденного unit-тестирования}
\imgh{150mm}{screenshot004}{Сравнение с онлайн-калькулятором}
\section{Приложение}
\subsection{Deposit.java - основной класс}
\lstinputlisting[language=Java, caption=Основной класс]{Deposit.java}
\subsection{Test.java - unit-тесты}
\lstinputlisting[language=Java, caption=UNIT-тесты]{Test.java}
\subsection{Proc.java - энумератор}
\lstinputlisting[language=Java, caption=ENUM для текстбокса]{Proc.java}
\subsection{wind.java - GUI и ввод}
\lstinputlisting[language=Java, caption=GUI и обработка ввода]{wind.java}

